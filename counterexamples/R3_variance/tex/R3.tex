\documentclass{article}%
\usepackage[T1]{fontenc}%
\usepackage[utf8]{inputenc}%
\usepackage{lmodern}%
\usepackage{textcomp}%
\usepackage{lastpage}%
\usepackage{geometry}%
\usepackage{tabularx}%
\usepackage{booktabs}%
\usepackage[dvipsnames]{xcolor}
\usepackage{tikz}
\usepackage{tkz-graph}
\usepackage{tkz-berge}
\usetikzlibrary{arrows,shapes}
\usepackage[matrix,arrow,curve,cmtip]{xy}
\usepackage{svg}
\usepackage{multicol}
\usepackage{float}
\usepackage{graphicx}
\usepackage[shortlabels]{enumitem}
\usepackage{hyperref}
\usepackage[nottoc,numbib]{tocbibind}
\geometry{tmargin=2cm,lmargin=2cm,rmargin=2cm,bmargin=2cm}%
%
%
%
\definecolor{red0}{rgb}{1.0,0,0}
\definecolor{green0}{rgb}{0,1.0,0}
%
%
%
\pdfsuppresswarningpagegroup=1

\setcounter{tocdepth}{2}

\title{Pinning poset changes drastically under Reidemeister III}

\author{Christopher-Lloyd Simon and Ben Stucky}

\begin{document}%

\section{Pinning poset changes drastically under Reidemeister III}

\subsubsection{[(1, 7, 2, 6), (24, 7, 1, 8), (5, 10, 6, 11), (4, 12, 5, 11), (9, 15, 10, 14), (8, 15, 9, 16), (13, 18, 14, 19), (12, 20, 13, 19), (2, 22, 3, 21), (3, 20, 4, 21), (17, 23, 18, 22), (16, 23, 17, 24)]}

{\small\noindent PD code drawn by \texttt{SnapPy}: [(3, 22, 4, 23), (21, 4, 22, 5), (2, 7, 3, 8), (8, 1, 9, 2), (11, 6, 12, 7), (5, 12, 6, 13), (10, 15, 11, 16), (16, 9, 17, 10), (24, 17, 1, 18), (18, 23, 19, 24), (19, 14, 20, 15), (13, 20, 14, 21)]}

{\small\noindent Planar representation generated by \texttt{plantri}: -}

\begin{multicols}{2}
{\normalsize \noindent\textbf{Total optimal pinning sets:} 1

\noindent\textbf{Total minimal pinning sets:} 1

\noindent\textbf{Total pinning sets:} 256

\noindent\textbf{Pinning number:} 6

}
\columnbreak

{\normalsize \noindent\textbf{Average optimal degree:} 2.0

\noindent\textbf{Average minimal degree:} 2.0

\noindent\textbf{Average overall degree:} 2.97

}
\end{multicols}

\begin{table}[ht]
	\caption{Pinning sets/average degree by cardinal}
	\centering
	\renewcommand{\arraystretch}{1.5}
	\begin{tabularx}{\textwidth}{lXXXXXXXXXXX}
		\toprule
			Cardinal & 6 & 7 & 8 & 9 & 10 & 11 & 12 & 13 & 14 & Total\\
			\hline
			Optimal pinning sets & 1 & 0 & 0 & 0 & 0 & 0 & 0 & 0 & 0 & 1 \\
			Minimal (suboptimal) pinning sets & 0 & 0 & 0 & 0 & 0 & 0 & 0 & 0 & 0 & 0 \\
			Nonminimal pinning sets & 0 & 8 & 28 & 56 & 70 & 56 & 28 & 8 & 1 & 255 \\
			Average degree & 2.0 & 2.36 & 2.62 & 2.83 & 3.0 & 3.14 & 3.25 & 3.35 & 3.43 &  \\
		\bottomrule \\ 
	\end{tabularx}
\end{table}

\begin{multicols}{2}
\begin{figure}[H]
\centering
\includesvg[inkscapelatex=false,width=250pt]{tex/img/[(1, 7, 2, 6), (24, 7, 1, 8), (5, 10, 6, 11), (4, 12, 5, 11), (9, 15, 10, 14), (8, 15, 9, 16), (13, 18, 14, 19), (12, 20, 13, 19), (2, 22, 3, 21), (3, 20, 4, 21), (17, 23, 18, 22), (16, 23, 17, 24)].svg}
\caption{\texttt{SnapPy} multiloop plot.}
\label{fig:tex/img/[(1, 7, 2, 6), (24, 7, 1, 8), (5, 10, 6, 11), (4, 12, 5, 11), (9, 15, 10, 14), (8, 15, 9, 16), (13, 18, 14, 19), (12, 20, 13, 19), (2, 22, 3, 21), (3, 20, 4, 21), (17, 23, 18, 22), (16, 23, 17, 24)].svg}
\end{figure}
\columnbreak

\begin{figure}[H]
\centering
\scalebox{0.8}{\input{tex/img/[(1, 7, 2, 6), (24, 7, 1, 8), (5, 10, 6, 11), (4, 12, 5, 11), (9, 15, 10, 14), (8, 15, 9, 16), (13, 18, 14, 19), (12, 20, 13, 19), (2, 22, 3, 21), (3, 20, 4, 21), (17, 23, 18, 22), (16, 23, 17, 24)].pgf}}
\caption{Minimal join sub-semi-lattice of minimal pinning sets.}
\label{fig:tex/img/[(1, 7, 2, 6), (24, 7, 1, 8), (5, 10, 6, 11), (4, 12, 5, 11), (9, 15, 10, 14), (8, 15, 9, 16), (13, 18, 14, 19), (12, 20, 13, 19), (2, 22, 3, 21), (3, 20, 4, 21), (17, 23, 18, 22), (16, 23, 17, 24)].pgf}
\end{figure}
\end{multicols}

\newpage

\subsubsection{[(1, 7, 2, 6), (24, 7, 1, 8), (5, 10, 6, 11), (3, 13, 4, 12), (9, 15, 10, 14), (8, 15, 9, 16), (13, 18, 14, 19), (4, 19, 5, 20), (2, 22, 3, 21), (11, 21, 12, 20), (17, 23, 18, 22), (16, 23, 17, 24)]}

{\small\noindent PD code drawn by \texttt{SnapPy}: [(4, 23, 5, 24), (22, 5, 23, 6), (3, 8, 4, 9), (10, 1, 11, 2), (12, 7, 13, 8), (6, 13, 7, 14), (11, 16, 12, 17), (2, 17, 3, 18), (18, 9, 19, 10), (19, 24, 20, 1), (20, 15, 21, 16), (14, 21, 15, 22)]}

{\small\noindent Planar representation generated by \texttt{plantri}: -}

\begin{multicols}{2}
{\normalsize \noindent\textbf{Total optimal pinning sets:} 1

\noindent\textbf{Total minimal pinning sets:} 2

\noindent\textbf{Total pinning sets:} 1152

\noindent\textbf{Pinning number:} 4

}
\columnbreak

{\normalsize \noindent\textbf{Average optimal degree:} 2.25

\noindent\textbf{Average minimal degree:} 2.38

\noindent\textbf{Average overall degree:} 3.14

}
\end{multicols}

\begin{table}[ht]
	\caption{Pinning sets/average degree by cardinal}
	\centering
	\renewcommand{\arraystretch}{1.5}
	\begin{tabularx}{\textwidth}{lXXXXXXXXXXXXX}
		\toprule
			Cardinal & 4 & 5 & 6 & 7 & 8 & 9 & 10 & 11 & 12 & 13 & 14 & Total\\
			\hline
			Optimal pinning sets & 1 & 0 & 0 & 0 & 0 & 0 & 0 & 0 & 0 & 0 & 0 & 1 \\
			Minimal (suboptimal) pinning sets & 0 & 0 & 1 & 0 & 0 & 0 & 0 & 0 & 0 & 0 & 0 & 1 \\
			Nonminimal pinning sets & 0 & 10 & 45 & 127 & 231 & 287 & 245 & 141 & 52 & 11 & 1 & 1150 \\
			Average degree & 2.25 & 2.58 & 2.79 & 2.95 & 3.06 & 3.16 & 3.24 & 3.3 & 3.36 & 3.4 & 3.43 &  \\
		\bottomrule \\ 
	\end{tabularx}
\end{table}

\begin{multicols}{2}
\begin{figure}[H]
\centering
\includesvg[inkscapelatex=false,width=250pt]{tex/img/[(1, 7, 2, 6), (24, 7, 1, 8), (5, 10, 6, 11), (3, 13, 4, 12), (9, 15, 10, 14), (8, 15, 9, 16), (13, 18, 14, 19), (4, 19, 5, 20), (2, 22, 3, 21), (11, 21, 12, 20), (17, 23, 18, 22), (16, 23, 17, 24)].svg}
\caption{\texttt{SnapPy} multiloop plot.}
\label{fig:tex/img/[(1, 7, 2, 6), (24, 7, 1, 8), (5, 10, 6, 11), (3, 13, 4, 12), (9, 15, 10, 14), (8, 15, 9, 16), (13, 18, 14, 19), (4, 19, 5, 20), (2, 22, 3, 21), (11, 21, 12, 20), (17, 23, 18, 22), (16, 23, 17, 24)].svg}
\end{figure}
\columnbreak

\begin{figure}[H]
\centering
\scalebox{0.8}{\input{tex/img/[(1, 7, 2, 6), (24, 7, 1, 8), (5, 10, 6, 11), (3, 13, 4, 12), (9, 15, 10, 14), (8, 15, 9, 16), (13, 18, 14, 19), (4, 19, 5, 20), (2, 22, 3, 21), (11, 21, 12, 20), (17, 23, 18, 22), (16, 23, 17, 24)].pgf}}
\caption{Minimal join sub-semi-lattice of minimal pinning sets.}
\label{fig:tex/img/[(1, 7, 2, 6), (24, 7, 1, 8), (5, 10, 6, 11), (3, 13, 4, 12), (9, 15, 10, 14), (8, 15, 9, 16), (13, 18, 14, 19), (4, 19, 5, 20), (2, 22, 3, 21), (11, 21, 12, 20), (17, 23, 18, 22), (16, 23, 17, 24)].pgf}
\end{figure}
\end{multicols}

\newpage


\end{document}